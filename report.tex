\documentclass[a4paper,12pt,titlepage,finall]{article}

\usepackage[T1,T2A]{fontenc}     % форматы шрифтов
\usepackage[utf8x]{inputenc}     % кодировка символов, используемая в данном файле
\usepackage{graphicx}
\graphicspath{{./}}
\usepackage{blindtext}
\usepackage[russian]{babel}      % пакет русификации
\usepackage{tikz}                % для создания иллюстраций
\usepackage{pgfplots}            % для вывода графиков функций
\usepackage{geometry}		 % для настройки размера полей
\usepackage{indentfirst}         % для отступа в первом абзаце секции
\usepackage{listings}
% выбираем размер листа А4, все поля ставим по 3см
\geometry{a4paper,left=30mm,top=30mm,bottom=30mm,right=30mm}

\setcounter{secnumdepth}{0}      % отключаем нумерацию секций

\usepgfplotslibrary{fillbetween} % для изображения областей на графиках

\begin{document}
% Титульный лист
\begin{titlepage}
    \begin{center}
	{\small \sc Московский государственный университет \\имени М.~В.~Ломоносова\\
	Факультет вычислительной математики и кибернетики\\}
	\vfill
	{\Large \sc Отчет по заданию №6}\\
	~\\
	{\large \bf <<Сборка многомодульных программ. \\
	Вычисление корней уравнений и определенных интегралов.>>}\\ 
	~\\
	{\large \bf Вариант 10 / 3 / 3}
    \end{center}
    \begin{flushright}
	\vfill {Выполнил:\\
	студент 106 группы\\
	Якунин В. П.\\
	~\\
	Преподаватели:\\
	Корухова Л. С.\\
    Манушин Д. В.}
    \end{flushright}
    \begin{center}
	\vfill
	{\small Москва\\2023}
    \end{center}
\end{titlepage}

% Автоматически генерируем оглавление на отдельной странице
\tableofcontents
\newpage

\section{Постановка задачи}

В задании требуется с заданной точностью $\varepsilon = 10^{-3}$ вычислить площадь плоской фигуры, ограниченной тремя кривыми, заданными следующими функциями:
\[
f_1 = 1 + \frac{4}{x^2 + 1} \hspace{1cm}
f_2 = x^3 \hspace{1cm}
f_3 = 2^{-x}
\]
Площадь вычисляется как сумма площадей под графиками функций на соответствующих отрезках, ограниченных точками пересечения графиков функций.\\ Для нахождения площади под графиком на заданном отрезке воспользоваться одним из методов численного интегрирования: методом Симпсона(метод парабол).\\ Для поиска точек пересечения графиков функций $f(x)$ и $g(x)$ требуется рассмотреть промежуточную функцию $F(x) = f(x) - g(x)$ и найти её корень на выбранном отрезке методом Ньютона (метод касательных).\\ Отрезки, на которых мы ищем корень, определены заранее вручную при анализе исходных функций.

\newpage

\section{Математическое обоснование}

Для того, чтобы можно было применять метод касательных надо, чтобы на отрезке, где он применяется, первая и вторая производная функции $F(x) = f(x) - g(x)$ не меняли знака:
\[
f_1' = -8 \frac{x}{x^4 + 2x^2 + 1} \quad f_2' = 3x^2 \quad f_3'= -2^{-x}ln2
\]
\[
f_1'' = \frac{24x^2 - 8}{x^6 + 3x^4 + 3x^2 + 1} \quad f_2'' = 6x \quad f_3'' = 2^{-x}ln^22
\]

производная $F'(x) = f'(x) - g'(x)$, и $F''(x) = f''(x) - g''(x)$. Построив графики производных и вторых производных определим отрезки, на которых знак не меняется: для $f_1, f_2 - [0.5, 2]$ , $f_1, f_3 - [-1.4, -0.7]$ , $f_3, f_2 - [0.5, 1.5]$ .\\ Для того, чтобы при суммировании площадей значение было вычисленно с точность $\varepsilon = 10^{-3}$ требуется, чтобы сами площади были вычислены с точностью, на один знак большей, чтобы избежать погрешности вычислений с плавающей точкой, поэтому $\varepsilon_2 = 10^{-4}$.\\ А из-за погрешности вычислений с плавающей точкой при расчёте значений функции в точке точность вычисления корня должна быть ещё на один порядок выше, поэтому $\varepsilon_1 = 10^{-5}$. При сравнении ответа, полученного своей программой с ответом от готовых матпакетов, таких как wolfram alpha оказалось, что конкретно для данных функций ответ получен даже с более высокой точностью и расходится в шестом знаке после запятой, однако уменьшение точности вычислений увеличением $\varepsilon_1$ или $\varepsilon_2$ всего на один порядок приводит резкому уменьшению точности ответа. \\

Ограничение на сохранение знака второй и первой производных на отрезке необходимо для того, чтобы работал метод Ньютона, без этого иттерационная последовательность, которую мы строим в методе и предел которой мы приближённо ищем, не обязана сходится \cite{math}.\\

При численном интегрировании условием окончания вычисления интегральных сумм является неравенство $h|I_{2n} - I_n| < \varepsilon_2$, где $I_n$ и  $I_{2n}$ --- старая и новая интегральные суммы. Это следует из правила Рунге \cite{Rhunge} : $h|I_{2n} - I_n| \cong I_n - I$, где $h = \frac{1}{15}$ --- константа, зависящая от метода интегрирования(так, для метода прямоугольников она равнялась бы $\frac{1}{3}$), а $I$ --- искомое значение интергала.
 
\begin{figure}[h]
\centering
\begin{tikzpicture}
\begin{axis}[% grid=both,                % рисуем координатную сетку (если нужно)
             axis lines=middle,          % рисуем оси координат в привычном для математики месте
             restrict x to domain=-2:4,  % задаем диапазон значений переменной x
             restrict y to domain=-1:6,  % задаем диапазон значений функции y(x)
             axis equal,                 % требуем соблюдения пропорций по осям x и y
             enlargelimits,              % разрешаем при необходимости увеличивать диапазоны переменных
             legend cell align=left,     % задаем выравнивание в рамке обозначений
             scale=2]                    % задаем масштаб 2:1

% первая функция
% параметр samples отвечает за качество прорисовки
\addplot[green,samples=256,thick] {1 + 4/(x^2 + 1)};
% описание первой функции
\addlegendentry{$y=1 + \frac{4}{x^2 + 1}$}

% добавим немного пустого места между описанием первой и второй функций
\addlegendimage{empty legend}\addlegendentry{}

% вторая функция
% здесь необходимо дополнительно ограничить диапазон значений переменной x
\addplot[blue,domain=-0.5:4,samples=256,thick] {x^3};
\addlegendentry{$y=x^3$}

% дополнительное пустое место не требуется, так как формулы имеют небольшой размер по высоте

% третья функция
\addplot[red,samples=256,thick] {exp(-x * ln(2)};
\addlegendentry{$y=2^{-x}$}
\end{axis}
\end{tikzpicture}
\caption{Плоская фигура, ограниченная графиками заданных уравнений}
\label{plot1}
\end{figure}


\newpage

\section{Результаты экспериментов}

Координаты точек пересечения, посчитанные методом касательных:
\begin{table}[h]
\centering
\begin{tabular}{|c|c|c|}
\hline
Кривые & $x$ & $y$ \\
\hline
1 и 2 &  1.3437 & 2.4261 \\
2 и 3 &  0.8262 & 0.5639 \\
1 и 3 & -1.3078 & 2.4756 \\
\hline
\end{tabular}
\caption{Координаты точек пересечения}
\label{table1}
\end{table}

Отрезки интегрирования и значения площадей под графиками на данных отрезках:

\begin{table}[h]
    \centering
    \begin{tabular}{|c|c|c|c|}
    \hline
         кривая & начало отрезка & конец отрезка & площадь\\
         \hline
         1 & -1.3078 & 1.3437 & 10.0477\\
         2 & 0.8262 & 1.3437 & 0.6985\\
         3 & -1.3078 & 0.8262 & 2.7580\\
         \hline
    \end{tabular}
    \caption{значения площадей под графиками}
    \label{table2}
\end{table}

Ответ на задачу вычисляется как $S = S_1 - S_2 - S_3 = 10.0477 - 0.6985 - 2.7580 = 6.5912$. Так как ответ просят получить с точностью до 3 знаков после запятой, то $S \approx 6.591$
\begin{figure}[h]
\centering
\begin{tikzpicture}
\begin{axis}[% grid=both,                % рисуем координатную сетку (если нужно)
             axis lines=middle,          % рисуем оси координат в привычном для математики месте
             restrict x to domain=-2:4,  % задаем диапазон значений переменной x
             restrict y to domain=-1:6,  % задаем диапазон значений функции y(x)
             axis equal,                 % требуем соблюдения пропорций по осям x и y
             enlargelimits,              % разрешаем при необходимости увеличивать диапазоны переменных
             legend cell align=left,     % задаем выравнивание в рамке обозначений
             scale=2,                    % задаем масштаб 2:1
             xticklabels={,,},           % убираем нумерацию с оси x
             yticklabels={,,}]           % убираем нумерацию с оси y

% первая функция
% параметр samples отвечает за качество прорисовки
\addplot[green,samples=256,thick,name path=A] {1 + 4/(x^2 + 1)};
% описание первой функции
\addlegendentry{$y=1 + \frac{4}{x^2 + 1}$}

% добавим немного пустого места между описанием первой и второй функций
\addlegendimage{empty legend}\addlegendentry{}

% вторая функция
% здесь необходимо дополнительно ограничить диапазон значений переменной x
\addplot[blue,domain=-0.5:4,samples=256,thick,name path=B] {x^3};
\addlegendentry{$y=x^3$}

% дополнительное пустое место не требуется, так как формулы имеют небольшой размер по высоте

% третья функция
\addplot[red,samples=256,thick,name path=C] {exp(-x * ln(2)};
\addlegendentry{$y=2^{-x}$}

% закрашиваем фигуру

\addplot[blue!20,samples=256] fill between[of=A and C,soft clip={domain=-1.3078:0.8262}];
\addplot[blue!20,samples=256] fill between[of=A and B,soft clip={domain=0.8:1.3437}];

\addlegendentry{$S=6.591$}

% Поскольку автоматическое вычисление точек пересечения кривых в TiKZ реализовать сложно,
% будем явно задавать координаты.
\addplot[dashed] coordinates { (-1.3078, 2.4756) (-1.3078, 0) };
\addplot[color=black] coordinates {(-1.3078, 0)} node [label={-90:{\small -1.3078}}]{};

\addplot[dashed] coordinates { (0.8262, 0.5639) (0.8262, 0) };
\addplot[color=black] coordinates {(0.8262, 0)} node [label={-100:{\small 0.8262}}]{};

\addplot[dashed] coordinates { (1.3437, 2.4261) (1.3437, 0) };
\addplot[color=black] coordinates {(1.3437, 0)} node [label={-90:{\small 1.3437}}]{};

\end{axis}
\end{tikzpicture}
\caption{Плоская фигура, ограниченная графиками заданных уравнений}
\label{plot2}
\end{figure}

\newpage

\section{Структура программы и спецификация функций}

Программа состоит из файлов \textsl{source.c} и \textsl{functions.asm}, каждый из которых содержит следующие модули:\\

\textsl{source.c}: содежит функции \textsl{root} и \textsl{integral}, а также основную функцию \textsl{main}, набот математических функций для отладки и функции для печати ответа для всех аргументов командной строки:\\



\textsl{double root ( double (*f)(double), double (* fder)(double), double (*g)(double), double (* gder)(double), double a, double b, double eps1)} --- функция нахождения точки пересечения двух графиков функций методом касательных. Принимает указатели на две функции и на их производные, а также границы отрезка, на котором производится поиск и точность, с которой будет найден корень. Возвращает найденное значение корня с точностью \textsl{eps1}.\\

\textsl{double integral ( double (*f)(double), double a, double b, double eps2)} --- функция вычисления определенного интеграла заданной функции на отрезке $[a, b]$ с точностью \textsl{eps2}. Принимает указатель на функцию, границы итрезка интегрирования и точность, с которой происходит интегрирование. Функция возвращает значение интеграла на отрезке.\\

Глобальная переменная \textsl{count} считает число иттераций при вычислении корня. Значения $\varepsilon_1$ и $\varepsilon_2$ тоже заданы как глобальные переменные.\\

\textsl{functions.asm} --- содержит реализации функций $f_1, f_2, f_3$ из задания и их производных.

Программа запускается из консоли и поддерживает следующие флаги:

\begin{itemize}
    \item \textsl{-help} ---  выводит информацию о командах и функциях.\\
    
    \includegraphics[scale=0.45]{help.png}
    \item \textsl{-roots} --- выводит абсциссы точек пересечения функций из задания.\\
    
    \includegraphics[]{roots.png}
    \item \textsl{-integrals} --- выводит значения интегралов функций из задания на промежутках, ограниченных соответствующими точками пересечения.\\
    
    \includegraphics[]{integrals.png}
    \item \textsl{-answer} --- выводит ответ на задачу.\\
    
    \includegraphics[]{answer.png}
    \item \textsl{-iters <номер первой функции> <номер второй функции> <левая граница отрезка> <правая граница отрезка>} --- считает число итераций при вычислении точки пересечения двух выбранных функций на выбранном интервале.\\

    \includegraphics[]{iters.png}
    \item \textsl{-test-roots <номер первой функции> <номер второй функции> <левая граница отрезка> <правая граница отрезка>} --- находит точку пересечения двух выбранных функций на выбранном интервале.\\

    \includegraphics[]{test_roots.png}
    \item \textsl{-test-integrals <номер функции> <левая граница отрезка> <правая граница отрезка>} --- находит интеграл выбранной функции на выбранном отрезке.\\

    \includegraphics[]{test_integrals.png}
\end{itemize}

Если вводится неправильный ключ, то программа сообщит вам об этом:\\

\includegraphics[]{incorrect.png}

Подерживается воод нескольких ключей подряд:\\

\includegraphics[scale=0.45]{all.png}

\newpage

\section{Сборка программы (Make-файл)}
\begin{center}
    текст Make-файла
\end{center}

\begin{lstlisting}
all: integral

source.o : source.c
	gcc -m32 -c -o source.o source.c 
functions.o : functions.asm
	nasm -f elf32 -o functions.o functions.asm
integral: source.o functions.o
	gcc -o integral source.o functions.o -m32 -lm 
	
clean :
	rm -rf source.o functions.o 
\end{lstlisting}

Схема работы компановщика (также на последнем шаге подключается math.h).\\

\includegraphics[scale=0.7]{diagram.png}

\newpage

\section{Отладка программы, тестирование функций}

Отладка \textsl{root}:

будем тестировать на следующих функциях, чьи точки пересечения легко увидеть либо посчитать вручную, корни ищем на отрезке $[0.5, 2]$ для первых двух функций и на отрезке $[-1, 1]$ для третьей:\\

\begin{tabular}{|c|c|c|c|}
    \hline
     функция №1 & функция №2 & ответ программы & аналитический ответ \\
     \hline
     $y = x$ & $y = 1$ & 1.0000 & 1\\
     $y = x$ & $y = 4 - x^2$ & 1.561553 & $\frac{\sqrt{17} - 1}{2}=$ 1.561552812...\\
     $y = 1$ & $y = 2^{-x}$ &  0.000000 & 0\\
     \hline
\end{tabular}\\

тестирование \textsl{integral}:
будем вычислять определенный интеграл функции $f$ на интервале $[a, b]$ и сравнивать с полученным аналитически.\\

\begin{tabular}{|c|c|c|c|c|c|}
    \hline
     функция $f$ & интеграл & a & b & ответ программы & аналитический ответ \\
     \hline
     $y = 1$& $y = x$ & -4 & 4 & 8.0000 & 8\\
     $y = x^3$ & $\frac{x^4}{4}$ & 0 & 4 & 64.0000 & 64\\
     $y = 4 - x^2$& $4x - \frac{x^3}{3}$ & 0 & 3 & 3.00000 & 3\\
     \hline
\end{tabular}

\newpage

\section{Программа на Си и на Ассемблере}


Исходные тексты программы приложены к этому отчету.


\newpage

\section{Анализ допущенных ошибок}

\begin{enumerate}
    \item Сохранял промежуточные значения в функции \textsl{integral}, что приводило к чрезмерному расходу памяти. Исправлено.\\
    \item Вместо // для комментария написал / из-за чего программа не работала. Исправлено.\\
    \item Неправильно записал иструкции в Make-файл, из-за чето вся программа пересобиралась заново при использовании команды make. Исправлено.\\
\end{enumerate}

\newpage
\begin{raggedright}
\addcontentsline{toc}{section}{Список цитируемой литературы}
\begin{thebibliography}{99}
\bibitem{math} Ильин~В.\,А., Садовничий~В.\,А., Сендов~Бл.\,Х. Математический анализ. Т.\,1~---
    Москва: Наука, 1985.
\bibitem{Rhunge} «Задания практикума на ЭВМ» Трифонов Н.П., Пильщиков В.Н
\end{thebibliography}
\textbf{Список использованных сайтов:}\\
\textit{www.geogebra.org}\\
\textit{www.wolframalpha.com}\\
\end{raggedright}

\newpage

\end{document}

